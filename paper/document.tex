\documentclass{article}

% To improve reproducibility, don't include certain metadata generated from the
% system date and time.
\pdfvariable suppressoptionalinfo \numexpr32+64+512\relax

% Language setting
\usepackage[english]{babel}

% Set page size and margins
\usepackage[a4paper,top=2cm,bottom=2cm,left=3cm,right=3cm,marginparwidth=1.75cm]{geometry}
\usepackage{graphicx}
\usepackage{amsmath}
\usepackage{hyperref}
\usepackage{geometry}
\usepackage{csquotes}
\usepackage[numbers]{natbib}
\usepackage{float}

% Set title, authr and date
\title{\textbf{Migraines and Math Degrees: A Spurious Love Story}}
\author{Researchers with a Sense of Humor}
\date{May 28, 2025}


\begin{document}
\maketitle

\begin{abstract}
  In this groundbreaking investigation, we explore a striking and absolutely questionable correlation between the number of Master's degrees awarded in Mathematics and Statistics, and the frequency of Google searches for the phrase ``why do I have a migraine'' from 2012 to 2021. While the relationship is likely spurious, the implications for public health, education, and the cosmic balance of the universe are clearly profound.
\end{abstract}

\begin{section}{Introduction}
 Correlations are everywhere---sometimes meaningful, other times hilariously coincidental. Inspired by the timeless wisdom of tylervigen.com, we delved into a curious trend: as the number of Master's degrees awarded in Mathematics and Statistics increases, so too does public interest in migraines, as measured by Google search queries \cite{vigen}.
\end{section}

\begin{section}{Data and Methods}
 We utilized data featured by Tyler Vigen, drawn from publicly available sources:
 \begin{itemize}
   \item \textbf{Master's degrees awarded in Mathematics and Statistics}: National Center for Education Statistics. \cite{nc_es}
   \item \textbf{Google searches for ``why do I have a migraine''}: Google Trends. \cite{google_trends}
 \end{itemize}
 The analysis covered the years 2012 through 2021. Pearson's correlation coefficient was used to measure the linear relationship, yielding a stunning correlation of $r =$ \input{generated/correlation_value.tex}, with $r^2 =$ \input{generated/r_squared_value.tex} and $p$-value of \input{generated/p_value.tex}.
\end{section}

\begin{section}{Results}
 The correlation coefficient of \input{generated/correlation_value.tex} suggests an almost comically strong linear relationship. Approximately \input{generated/r_squared_percentage_value.tex} of the variation in migraine search frequency could be ``explained'' by math degree trends. The data paints a picture of synchronized suffering.

 \begin{figure}[H]
   \centering
   \includegraphics[width=0.6\textwidth]{./generated/correlation_plot.png}
   \caption{Spurious correlation between Master's degrees in Math/Stats and Google searches for ``why do I have a migraine''}
   \label{fig:correlation}
 \end{figure}

\end{section}

\begin{section}{Discussion}
 Several hypotheses could explain this correlation:
 \begin{itemize}
   \item Pursuing a Master's degree in Mathematics and Statistics induces migraines.
   \item Sufferers of migraines seek solace in rigorous mathematical study.
   \item This is all an elaborate prank by the universe.
 \end{itemize}
 Clearly, none of these should be taken seriously. Except maybe the last one.
\end{section}

\begin{section}{Conclusion}
 Is this correlation real? Yes. Is it meaningful? No. Should you change careers or question your reality because of it? Absolutely. But more than anything, remember: correlation is not causation—but it sure is fun.
\end{section}

\begin{section}{Acknowledgments}
 We thank Tyler Vigen for inspiring this research and providing sources. We also thank OpenAI for providing the tools to generate this paper, and our families and friends for pretending to understand what we do.
 We also want to thank our professors for allowing us to write this paper instead of a real one, and not failing us for it. Finally, we thank the universe for its endless supply of spurious correlations that keep us entertained.
\end{section}


\begin{thebibliography}{9}

  \bibitem{vigen}
  Vigen, T. (2024, February 13). \textit{Master's degrees awarded in Mathematics and Statistics correlates with Google searches for ``why do I have a migraine''}. Retrieved June 2, 2025, from \url{https://web.archive.org/web/20250602220936/https://www.tylervigen.com/spurious/correlation/33216_masters-degrees-awarded-in-mathematics-and-statistics_correlates-with_google-searches-for-why-do-i-have-a-migraine}

  \bibitem{nc_es}
  National Center for Education Statistics. (2024). \textit{Degrees conferred by degree-granting postsecondary institutions, by level of degree and field of study: Selected years, 2012 through 2021}. Retrieved June 2, 2025, from \url{https://nces.ed.gov/programs/digest/d21/tables/dt21_322.10.asp}

  \bibitem{google_trends}
  Google Trends. (2024). \textit{Search interest in ``why do I have a migraine'' from 2012 to 2021}. Retrieved June 2, 2025, from \url{https://trends.google.com/trends/explore?date=2012-01-01%202021-12-31&q=why%20do%20I%20have%20a%20migraine}


\end{thebibliography}

\end{document}
